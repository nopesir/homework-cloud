\chapter{Final considerations}
In this last chapter are presented some observability topics, while a conclusion section will end this report.
\minitoc

\section{Observability}
An important concept in the world of microservices, is \textit{monitoring}. It is an action used against apps and systems in order to determine their state. From basic tests and whether they are running or not, to more proactive performance health checks. Apps are monitored to detect problems and anomalies. As troubleshooters, DevOps use it to find the root cause of problems and gain insights into capacity requirements and performance trends over time.

Monitoring is not enough though. It has also evolved to support many more stakeholders: during application development, developers use monitoring to correlate coding practices to performance outcomes, while architects can validate which cloud patterns and models deliver the most for the price.
To achieve all of this, monitoring tools use many smart techniques like instrumentation and tracing to gather, digest, correlate and analyze a large amount of metrics across modern application stacks under the DevOps watch. Additionally, there is synthetic transactions and application experience analytics to gain critical insights into the digital aimless of the customers.
Basically, as the definition states, it is a measure of how well internal states of a system can be inferred from knowledge of its external outputs. Therefore, in contrast to monitoring (which is something someone actually do), observability is more a property of a system.

Istio exploits observability automatically for all the service communications in the mesh, in order to simplify troubleshooting, maintainance and optimization of all the applications. This kind of telemetry are from both service-to-service and within the Istio components interactions.

The types of telemetry generated by Istio are 
\begin{itemize}
    \item \textit{metrics}: a way to monitor and understand the behavior of all the in/out/within traffic. 
    \item \textit{traces}: a way to monitor a request as it passes through the mesh.
    \item \textit{logs}: a way to monitor and understand the behaviour of a particular workload instance.
\end{itemize}

\noindent That support various back-end metrics services and release a great potential for the application development and debugging stages, especially in the security context. 

\section{Conclusions}
Istio is a challenging project that actually has hundreds of developers behind it. The project is a little bit green though, since the majority f the APIs are still in the alpha and beta stages. Nevertheless, the analyzed security mechanisms implemented are compliant to what is written in the documentation. Moreover, the community behind it is extremely capable, educated and ready to help: it does not seem an open source platform. Unfortunately, Istio changes are fast, a bit too fast to pledge stability (an essential feature for large scale usage and maintenance). As an example, in the last year the Istio Security architecture dramatically changed, forcing DevOps to extra work hours to run the so called "features migrations", for example from API alpa to API beta in the authorization policy context.

Finally, an important topic is the documentation of the code. There are no details concerning the implementation or at least the general description of each source file. It is true that there are hundreds of thousands of lines of code, but it could be helpful if someone is interested in analyzing the code or even in participating by design it. The community could help, but in this particular case can be very stressful to ask and ask and wait and ask, instead of directly read and learn and participate.